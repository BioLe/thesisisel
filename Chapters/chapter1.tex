% 
%  chapter1.tex
%  ThesisISEL
%  
%  Created by Matilde Pós-de-Mina Pato on 2012/10/09.
%
\chapter{Introduction}
\label{cha:introduction}

This package is distributed under GPLv3 License. If you have questions or \todo{A marginpar note!} doubts concerning the guarantees, rights and duties of those who use packages under GPLv3 License, please read \url{http://www.gnu.org/licenses/gpl.html}.
\todo[inline]{A a note in a line by itself.}


\textbf{Notas do autor}

A redação deve ser feita com frases curtas e objectivas, organizadas de acordo com a estrutura do trabalho, dando destaque a cada uma das partes abordadas, assim apresentadas: Introdução - Informar, em poucas palavras, o contexto em que o trabalho se insere, sintetizando a problemática estudada. Objectivo - Deve ser explicitado claramente. Métodos - Destacar os procedimentos metodológicos adoptados. Resultados - Destacar os mais relevantes para os objectivos pretendidos. Os trabalhos de natureza quantitativa devem apresentar resultados numéricos, assim como seu significado estatístico. Conclusões - Destacar as conclusões mais relevantes, os estudos adicionais recomendados e os pontos positivos e negativos que poderão influir no conhecimento. 

Para indexação da tese nas bases de dados e catálogos de bibliotecas devem ser apontados pelo autor as palavras-chave que identifiquem os assuntos nela tratados. Estes permitirão a recuperação da tese quando da busca da literatura publicada. 

\begin{center}
	\textbf{\large this package and template are not official for ISEL/IPL}.
\end{center}
