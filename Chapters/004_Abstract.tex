\abstractEN % Do NOT modify this line

Finding a new estate in a large city has increasingly become more complex, as people are both concerned with the estate itself and everything that surrounds it. The \textbf{15-Minute City} concept, addresses cities ensuring that their residents can fulfill six essential functions: home, work, commerce, health care, education, and entertainment --- all within a 15-minute radius, which varies according to the chosen travel mode.
To help people find properties that would fit their needs. According to this concept, we have developed a system that aims to provide an intuitive user interface that guides the user through the important decision of choosing a estate. System-wise, we built a reliable application following the microservices architecture guidelines, which future proofs our solution by segregating each part of the project and allowing it to easily scale. All the infrastructure is containerized on Docker and orchestrated with Kubernetes.
To present relevant information to the user, we had to gather data from three relevant sources, each with its unique challenges. The first source is estate data, extracted from Portuguese real estate agencies websites. In second city data, with relevant points of interest according to the six essential functions mentioned previously. And at last, the user data provided by the user itself to our profiling system which lets us understand his needs. 
With this application, we provide users a way to select the best estates including the surrounding environment, tailored to their day-to-day needs, something that, as far as we know, is a novelty in real estate agency websites. 

% Keywords of abstract in English
\begin{keywords}
15-Minute Cities, Cloud computing, Geographical systems, Distributed systems, Microservices.
\end{keywords} 
