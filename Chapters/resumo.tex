\abstractPT  % Do NOT modify this line

Independentemente da língua em que está escrita a dissertação, é necessário um resumo na língua do texto principal e um resumo noutra língua.  Assume-se que as duas línguas em questão serão sempre o Português e o Inglês.

O \emph{template} colocará automaticamente em primeiro lugar o resumo na língua do texto principal e depois o resumo na outra língua. 

Resumo é a versão precisa, sintética e selectiva do texto do documento, destacando os elementos de maior importância. O resumo possibilita a maior divulgação da tese e sua indexação em bases de dados.

A redação deve ser feita com frases curtas e objectivas, organizadas de acordo com a estrutura do trabalho, dando destaque a cada uma das partes abordadas, assim apresentadas: Introdução; Objectivo; Métodos ; Resultados  e Conclusões

O resumo não deve conter citações bibliográficas, tabelas, quadros, esquemas. 

E, deve-se evitar o uso de expressões como "O presente trabalho trata ...", "Nesta tese são discutidos....", "O documento conclui que....", "aparentemente é...." etc. 

Existe um limite de palavras, 300 palavras é o limite.

Para indexação da tese nas bases de dados e catálogos de bibliotecas devem ser apontados pelo autor as palavras-chave que identifiquem os assuntos nela tratados. Estes permitirão a recuperação da tese quando da busca da literatura publicada. 

% Keywords of abstract in Portuguese
\begin{keywords}
Palavras-chave (em português) \ldots
\end{keywords}
% to add an extra black line
