\abstractPT  % Do NOT modify this line

Encontrar uma casa numa grande cidade tornou-se um processo cada vez mais complexo, dado que as pessoas têm de se preocupar não só com o imóvel em si mas também com tudo o que o envolve. O conceito da cidade 15-minutos, imagina as cidades como uma forma de garantir que os seus cidadãos conseguem cumprir seis funções essenciais: casa, trabalho, comércio, saúde, educação, e entertenimento; tudo num raio de 15 minutos, que varia conforme o modo de deslocação escolhido.

Para ajudar as pessoas a encontrar uma propriedade que se encontre dentro dos ideais propostos por este conceito, construimos uma sistema que disponibiliza uma interface gráfica que guia o utilizador pelas decisões importantes de comprar uma casa. Em termos de sistema, construi-se uma aplicação confiável que segue as diretrizes da arquitetura de microserviços, que torna a aplicação à prova do futuro ao tornar cada secção do projecto independente e fácilmente escalável, em termos de produto e equipa. Toda esta infrastrutura é contentorizada com o \textit{Docker} e orquestrada com Kubernetes.

Foi necessário obter informação de três origens diferentes, todas com os seus desafios únicos. A primeira, dados de imóveis extraidos de páginas de agências imobiliárias locais. Em segundo dados da cidade, com pontos de interesse relevantes para as seis funções mencionadas anteriormente. E por fim, dados do utilizador providenciados pelo mesmo, durante o processo de criação do perfil, de forma a entender as suas necessidades.

Com isto é dado ao utilizador uma ferramenta para selecionar os melhores imóveis de acordo com o ambiente que as involve e sob medida, de acordo com as suas necessidades diárias, algo que até ao momento, é uma novidade em sites de imobiliário.

% Keywords of abstract in Portuguese
\begin{keywords}
Cidades 15-minutos, Computação em núvem, Sistemas geográficos, Sistemas distribuídos, Microserviços.
\end{keywords}
% to add an extra black line
