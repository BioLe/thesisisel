\def\thetitle{%
 [Título da dissertacao, do Projecto ou do Relatorio de Estágio]
}
\def\theauthor{%
 [Nome completo do autor]
}


%% Title of thesis
\title{ \thetitle }

%% Author(s) 
% gender: f for women or m for men and name
%  ------ Master report  (maximum of 1) ---------
\author[f]{\uppercase{\theauthor}}
\authordegree{(Grau do candidato)} 

%  ------ Bachelor report  (maximum of 3) ---------
%\author[f]{\uppercase{[Nome completo do primeiro autor]}}
%\author[f]{\uppercase{[Nome completo do segundo autor]}}
%\author[f]{\uppercase{[Nome completo do terceiro autor]}}
%% remove text inside { } for report document
%\authordegree{} 


%% Date
\themonth{MES}
\theyear{ANO}

%% Supervisors (maximum of 2)
% use [f] for female and [m] for male
%% ---------------------------------------------------%%
%% \adviser[m|f]{Category}{Name}
%% ---------------------------------------------------%%
\adviser[m]{[Nome do orientador]}{[Grau]}
\adviser[f]{[Nome do orientador]}{[Grau]}

%
%% Jury (maximum of 5 elements)
%% Use [p] for president, of the jury and [a] for referees
%% ---------------------------------------------------%%
%% \jury[p|a]{Category and Name}
%% ---------------------------------------------------%%
\jury[p]{[Grau e Nome do presidente do juri]}
%% Referees
\jury[a]{[Grau e  Nome do primeiro vogal]}
\jury[a]{[Grau e  Nome do segundo vogal]}
%\jury[a]{[Grau e  Nome do terceiro vogal]}
%\jury[a]{[Grau e  Nome do quarto vogal]}